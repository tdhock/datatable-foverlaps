\documentclass{beamer}
\usepackage{tikz}
\usepackage[all]{xy}
\usepackage{listings}
\usepackage{slashbox}
%\usepackage{booktabs}
\usepackage{amsmath,amssymb}
\usepackage{hyperref}
\usepackage{graphicx}

\DeclareMathOperator*{\argmin}{arg\,min}
\DeclareMathOperator*{\Lik}{Lik}
\DeclareMathOperator*{\Peaks}{Peaks}
\DeclareMathOperator*{\Segments}{Segments}
\DeclareMathOperator*{\argmax}{arg\,max}
\DeclareMathOperator*{\maximize}{maximize}
\DeclareMathOperator*{\minimize}{minimize}
\newcommand{\sign}{\operatorname{sign}}
\newcommand{\RR}{\mathbb R}
\newcommand{\ZZ}{\mathbb Z}
\newcommand{\NN}{\mathbb N}

% Set transparency of non-highlighted sections in the table of
% contents slide.
\setbeamertemplate{section in toc shaded}[default][100]
\AtBeginSection[]
{
  \setbeamercolor{section in toc}{fg=red} 
  \setbeamercolor{section in toc shaded}{fg=black} 
  \begin{frame}
    \tableofcontents[currentsection]
  \end{frame}
}

\begin{document}

\title{Is R package \texttt{data.table} really so fast?} 

\author{
  Toby Dylan Hocking\\
  toby.hocking@mail.mcgill.ca}

\date{12 February 2015}

\maketitle

\section{What is data.table?}

\begin{frame}
  \frametitle{data.table is an enhanced data.frame}
\end{frame}

\section{Benchmark timings on genomic data}

\begin{frame}
  \frametitle{data.table::fread faster than read.table for bedGraph
    files}
  \includegraphics[width=\textwidth]{figure-TF-benchmark}

  \texttt{read.table(file, sep=, colClasses=, nrows=)} vs
  \texttt{fread(file)}
\end{frame}

\begin{frame}
  \frametitle{data.table::foverlaps slower than
    GenomicRanges::findOverlaps}
  \includegraphics[width=\textwidth]{figure-TF-benchmark-overlap}

  Query is $\approx 450$ genomic windows.
\end{frame}

\end{document}
