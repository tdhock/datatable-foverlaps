\documentclass{beamer}
\usepackage{tikz}
\usepackage[all]{xy}
\usepackage{listings}
\usepackage{slashbox}
%\usepackage{booktabs}
\usepackage{amsmath,amssymb}
\usepackage{hyperref}
\usepackage{graphicx}

\DeclareMathOperator*{\argmin}{arg\,min}
\DeclareMathOperator*{\Lik}{Lik}
\DeclareMathOperator*{\Peaks}{Peaks}
\DeclareMathOperator*{\Segments}{Segments}
\DeclareMathOperator*{\argmax}{arg\,max}
\DeclareMathOperator*{\maximize}{maximize}
\DeclareMathOperator*{\minimize}{minimize}
\newcommand{\sign}{\operatorname{sign}}
\newcommand{\RR}{\mathbb R}
\newcommand{\ZZ}{\mathbb Z}
\newcommand{\NN}{\mathbb N}

% Set transparency of non-highlighted sections in the table of
% contents slide.
\setbeamertemplate{section in toc shaded}[default][100]
\AtBeginSection[]
{
  \setbeamercolor{section in toc}{fg=red} 
  \setbeamercolor{section in toc shaded}{fg=black} 
  \begin{frame}
    \tableofcontents[currentsection]
  \end{frame}
}

\begin{document}

\title{Is R package \texttt{data.table} really so fast?} 

\author{
  Toby Dylan Hocking\\
  toby.hocking@mail.mcgill.ca}

\date{12 February 2015}

\maketitle

\section{What is data.table?}

\begin{frame}
  \frametitle{Brief history of statistical computing}
  \begin{description}
  \item[1957] FORTRAN by John Backus (IBM).
  %\item[1966] SAS (North Carolina State Univ).
  \item[1976] S by John M Chambers (Bell Labs).
  %\item[1988] ``New S'' much rewritten in C.
  \item[1993] S exclusively licensed to StatSci/MathSoft.\\
    R by Ross Ihaka and Robert Gentleman\\
    (Univ Auckland, New Zealand).
  \item[1998] Association of Computing Machinery 
    Software System award
    to John M Chambers for
  \end{description}
  ``the S system, which has forever altered the way people analyze,
  visualize, and manipulate data.''
  
  \vskip 0.5cm Source: Wikipedia, R-FAQ.
\end{frame}

\begin{frame}
  \frametitle{R = interactive, graphical, programming with data}
  
  What is R? (Source: R-FAQ)

  ``R is a system for statistical computation and graphics. It
  consists of a language plus a run-time environment with graphics, a
  debugger, access to certain system functions, and the ability to run
  programs stored in script files.'' 

  \begin{description}
  \item[interactive] command line (versus compiled).
  \item[graphical] publication-quality plots.
  \item[programming with data] \texttt{data.frame} which represents a
    tabular data set.
  \end{description}
\end{frame}

\begin{frame}[fragile]
  \frametitle{data.table is an enhanced data.frame}

\begin{verbatim}
Author: M Dowle, T Short, S Lianoglou, A Srinivasan 
 with contributions from R Saporta, E Antonyan
Description: Fast aggregation of large data
 (e.g. 100GB in RAM), fast ordered joins, 
 fast add/modify/delete of columns 
 by group using no copies at all, 
 list columns and a fast file reader (fread). 
 Offers a natural and flexible syntax, 
 for faster development.
\end{verbatim}

\end{frame}

\section{Benchmark timings on genomic data}

\begin{frame}
  \frametitle{data.table::fread faster than read.table for bedGraph
    files}
  \includegraphics[width=\textwidth]{figure-TF-benchmark}

  \texttt{read.table(file, sep=, colClasses=, nrows=)} vs
  \texttt{fread(file)}
\end{frame}

\begin{frame}
  \frametitle{compute matching indices}
  \includegraphics[width=\textwidth]{figure-TF-benchmark-indices}

  $\approx 450$ genomic windows.
\end{frame}

\begin{frame}
  \frametitle{compute joined tables}
  \includegraphics[width=\textwidth]{figure-TF-benchmark-tables}

  $\approx 450$ genomic windows.
\end{frame}

\begin{frame}
  \frametitle{read, overlap, write}
  \includegraphics[width=\textwidth]{figure-TF-benchmark-IO}

  $\approx 450$ genomic windows.
\end{frame}

\begin{frame}
  \frametitle{data.table::foverlaps slower than
    GenomicRanges::findOverlaps????}
  \includegraphics[width=\textwidth]{figure-full-strand-times}

  413 genomic windows, size of bedGraph file in million rows.

\end{frame}

\begin{frame}[fragile]
  \frametitle{intersectBed -sorted is much slower}
  % thocking@silene:~/R/datatable-foverlaps(master*)$ time intersectBed -a max-windows.bed -b /home/thocking/genomecov/max/yale_k562_rep1.bedGraph-strand -sorted > overlap.bedGraph
% \begin{verbatim}
% thocking@silene:~/R/datatable-foverlaps(master*)$ time intersectBed -a max-windows.bed -b /home/thocking/genomecov/max/yale_k562_rep1.bedGraph-strand -sorted > overlap.bedGraph

% real	154m10.653s
% user	153m2.188s
% sys	0m20.204s
% thocking@silene:~/R/datatable-foverlaps(master)$ wc -l max-windows.bed  /home/thocking/genomecov/max/yale_k562_rep1.bedGraph-strand 
%       413 max-windows.bed
%   7617673 /home/thocking/genomecov/max/yale_k562_rep1.bedGraph-strand
% \end{verbatim}
\begin{verbatim}
$ time intersectBed -a windows.bed \
 -b chip-seq.bedGraph -sorted > overlap.bedGraph

real	154m10.653s
user	153m2.188s
sys	0m20.204s
$ wc -l windows.bed chip-seq.bedGraph overlap.bedGraph 
      413 windows.bed
  7617673 chip-seq.bedGraph
    47840 overlap.bedGraph
\end{verbatim}

% thocking@silene:~/R/datatable-foverlaps(master*)$ du -ms max-windows.bed  /home/thocking/genomecov/max/yale_k562_rep1.bedGraph-strand overlap.bedGraph 
% 1	max-windows.bed
% 189	/home/thocking/genomecov/max/yale_k562_rep1.bedGraph-strand
% 2	overlap.bedGraph

  
\end{frame}

\begin{frame}
  \frametitle{Conclusion: data.table really is fast!}
  For large ($\approx$ 10 million lines) genomic data files,
  \begin{itemize}
  \item \texttt{data.table::fread} faster than \texttt{read.table}.
  \item \texttt{data.table::foverlaps} faster than
    \texttt{GenomicRanges::findOverlaps}.
  \item \texttt{bedtools intersect} is slower than R.
  \end{itemize}

  Caveats:
  \begin{itemize}
  \item \texttt{data.table::fread} can't read gzip compressed files.
  \item \texttt{data.table::foverlaps} fastest with matching types
    (integer $\neq$ numeric).
  \end{itemize}
\end{frame}

\begin{frame}
  \frametitle{Thanks for your attention!}
  Any questions? \texttt{toby.hocking@mail.mcgill.ca}

  \vskip 1cm

  Source code for benchmarks is at
  \url{https://github.com/tdhock/datatable-foverlaps}
\end{frame}

\end{document}
